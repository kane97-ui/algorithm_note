\documentclass{article}
\usepackage[colorlinks, urlcolor=blue, linkcolor=red, citecolor=green]{hyperref}
\usepackage{fancyhdr} %设置页眉和页脚的
\usepackage{extramarks} %设置continue那玩意的
\usepackage{amsmath}
\usepackage{amsthm}
\usepackage{amsfonts}
\usepackage{tikz} %画线的
\usepackage[plain]{algorithm}
\usepackage{algpseudocode}
\usepackage{enumerate}
\usepackage{courier}
\usepackage{listings}
\usetikzlibrary{automata,positioning}

%表
\usepackage{booktabs}
\usepackage{multirow}
\usepackage{array}
\usepackage{caption}
\DeclareCaptionFont{heiti}{\heiti} %还可以定义其他的
\captionsetup{labelsep=space, font={small, bf}, skip=2pt} %space可以改成quad

%图
%*****************图片及其相关设置***************************
\usepackage{graphicx}
\graphicspath{{tupian/}}
\usepackage{subfigure}
% 导入tikz包
\usepackage{tikz}
\usetikzlibrary{math}

%*****************代码相关设置***************************
\usepackage{pythonhighlight}
%
% Basic Document Settings
%

\topmargin=-0.45in
\evensidemargin=0in
\oddsidemargin=0in
\textwidth=6.5in
\textheight=9.0in
\headsep=0.25in

\linespread{1.1}

\pagestyle{fancy}
\lhead{\hmwkAuthorName}
\chead{\hmwkClass: \hmwkTitle}
\rhead{\firstxmark}
\lfoot{\lastxmark}
\cfoot{\thepage}

\renewcommand\headrulewidth{0.4pt}
\renewcommand\footrulewidth{0.4pt}

\setlength\parindent{0pt}
\lstset{
 columns=fixed,       
 numbers=left,                                        % 在左侧显示行号
 numberstyle=\tiny\color{gray},                       % 设定行号格式
 frame=none,                                          % 不显示背景边框
 backgroundcolor=\color[RGB]{245,245,244},            % 设定背景颜色
 keywordstyle=\color[RGB]{40,40,255},                 % 设定关键字颜色
 numberstyle=\footnotesize\color{darkgray},           
 commentstyle=\it\color[RGB]{0,96,96},                % 设置代码注释的格式
 stringstyle=\rmfamily\slshape\color[RGB]{128,0,0},   % 设置字符串格式
 showstringspaces=false,                              % 不显示字符串中的空格
 language=c++,                                        % 设置语言
}

%
% Create Problem Sections
%

\newcommand{\enterProblemHeader}[1]{
    \nobreak\extramarks{}{Problem \arabic{#1} continued on next page\ldots}\nobreak{}
    \nobreak\extramarks{Problem \arabic{#1} (continued)}{Problem \arabic{#1} continued on next page\ldots}\nobreak{}
}

\newcommand{\exitProblemHeader}[1]{
    \nobreak\extramarks{Problem \arabic{#1} (continued)}{Problem \arabic{#1} continued on next page\ldots}\nobreak{}
    \stepcounter{#1}
    \nobreak\extramarks{Problem \arabic{#1}}{}\nobreak{}
}

\setcounter{secnumdepth}{0}
\newcounter{partCounter}
\newcounter{homeworkProblemCounter}
\setcounter{homeworkProblemCounter}{1}
\nobreak\extramarks{Problem \arabic{homeworkProblemCounter}}{}\nobreak{}

\newenvironment{homeworkProblem}{
    \section{Problem \arabic{homeworkProblemCounter}}
    \setcounter{partCounter}{1}
    \enterProblemHeader{homeworkProblemCounter}
}{
    \exitProblemHeader{homeworkProblemCounter}
}

%
% Homework Details
%   - Title
%   - Due date
%   - Class
%   - Section/Time
%   - Instructor
%   - Author
%

\newcommand{\hmwkTitle}{Homework\ \#2}
\newcommand{\hmwkDueDate}{Oct 28, 2021}
\newcommand{\hmwkClass}{DDA 6050}
\newcommand{\hmwkClassTime}{}
\newcommand{\hmwkClassInstructor}{Professor Yixiang Fang}
\newcommand{\hmwkAuthorName}{Haoyu Kang}
\newcommand{\hmwkAuthorSchool}{School of Data Science}
\newcommand{\hmwkAuthorNumber}{Sno.220041025}
%
% Title Page
%

\title{
    \vspace{2in}
    \textmd{\textbf{\hmwkClass:\ \hmwkTitle}}\\
    \normalsize\vspace{0.1in}\small{Due\ on\ \hmwkDueDate}\\
    \vspace{0.1in}\large{\textit{\hmwkClassInstructor\ \hmwkClassTime}}
    \vspace{3in}
}

\author{\textbf{\hmwkAuthorName}}


\date{}

\renewcommand{\part}[1]{\textbf{\large Part \Alph{partCounter}}\stepcounter{partCounter}\\}

%
% Various Helper Commands
%

% Useful for algorithms
\newcommand{\alg}[1]{\textsc{\bfseries \footnotesize #1}}
\usepackage[algo2e,vlined,ruled]{algorithm2e}

% For derivatives
\newcommand{\deriv}[1]{\frac{\mathrm{d}}{\mathrm{d}x} (#1)}

% For partial derivatives
\newcommand{\pderiv}[2]{\frac{\partial}{\partial #1} (#2)}

% Integral dx
\newcommand{\dx}{\mathrm{d}x}

% Alias for the Solution section header
\newcommand{\solution}{\textbf{\large Solution}}

% Probability commands: Expectation, Variance, Covariance, Bias
\newcommand{\E}{\mathrm{E}}
\newcommand{\Var}{\mathrm{Var}}
\newcommand{\Cov}{\mathrm{Cov}}
\newcommand{\Bias}{\mathrm{Bias}}
\begin{document}

\maketitle
\thispagestyle{empty}

\newpage
\setcounter{page}{1}

\begin{homeworkProblem}

\vspace{4pt}
\textbf{\large{LCS}}

\vspace{4pt}

The state transition equation is as below:
\begin{equation}
    \begin{split}
        c[i,j]=
        \left\{
        \begin{array}{lll}
                c[i-1,j-1]+1, \ \ x[i]=y[j] \\
                max(c[i-1,j],c[i,j-1]), \ \ otherwise.
        \end{array}
        \right.
    \end{split}
\end{equation}
For this question, we adopt method of state compression instead of a traditional approach that takes 
$O(n^2)$ memory. In details, we use \textbf{scrolling array} which is a one-demision array to save each states. 
For each inside loop, $dp[j]$ represents length of LCS between i-length of prefix A  and the j-length of prefix B.
The variable $temp$ will save the value of  LCS between i-1 prefix of A  and the j-1 prefix of B.
\vspace{4pt}

Hence the optimal method of LCS just takes \textbf{$O(n)$} space complexity and  \textbf{$O(n^2)$}
time complexity.

The cpp code is attached as follow:
\begin{lstlisting}
#include<iostream>
#include<vector>
#include<cstdio>
using namespace std;
int main(){
    int num;
    vector<int> seq1,seq2;
    while(scanf("%d",&num)) {
        seq1.push_back(num);
        if(cin.get()=='\n') break;
    }
    while(scanf("%d",&num)){
        seq2.push_back(num);
        if(cin.get()=='\n') break;
    }
    int n=seq1.size();
    vector<int> dp(n+1,0);
    int pre;
    for(int i=1;i<n+1;i++){
        for(int j=1;j<n+1;j++){
            int temp=dp[j];
            if(i==1 && j==1) dp[j]=seq1[i-1]==seq2[j-1]?1:0;
            else if(i==1)  dp[j]=seq1[i-1]==seq2[j-1]?1:dp[j-1];
            else if(j==1)  dp[j]=seq1[i-1]==seq2[j-1]?1:dp[j];
            else{
                if(seq1[i-1]==seq2[j-1]) dp[j]=pre+1;
                else dp[j]=max(dp[j-1],dp[j]);
            }
            pre=temp;
        }
    }
    cout<<dp[n]<<endl;
}
\end{lstlisting}


\vspace{4pt}
\end{homeworkProblem}
\begin{homeworkProblem}

\vspace{4pt}
\textbf{\large{0-1 Knapsack Problem}}

\vspace{4pt}

The state transition equation is as below:
\begin{equation}
    \begin{split}
        B[k,w]=
        \left\{
        \begin{array}{lll}
                B[k-1,w]+1, \ \ w_k>w \\
                max(B[k-1,w],B[k-1,w_k]+b_k), \ \ else.
        \end{array}
        \right.
    \end{split}
\end{equation}
We also adopt method of state compression instead of a traditional approach that takes 
$O(n^2)$ memory. In details, we use \textbf{scrolling array} which is a one-demision array to save each states. 
For each inside loop, $dp[j]$ represents maxmum value the bag is able to attain if the volume of bag is limited within j
respected to subset of $S_i$. Different from the problem 1, in this question we creat additional array to save states of the last once loop.
e.g. $pre[j-weight[i-1]]$ represents maxmum value the bag is able to attain if the volume of bag is limited within j-weight[i-1]
respected to subset of $S_{i-1}$\\

Hence the optimal method just takes \textbf{$O(n)$} space complexity and  \textbf{$O(n^2)$}
time complexity.\\

The cpp code is attached as follow:
\begin{lstlisting}
#include<iostream>
#include<vector>
using namespace std;
int main(){
    int n,w;
    cin>>n>>w;
    vector<int> weight,value;
    int w_i,v_i;
    for(int i=0;i<n;i++){
        cin>>w_i>>v_i;
        weight.push_back(w_i);
        value.push_back(v_i);
    }
    vector<int> dp(w+1,0);
    vector<int> pre(w+1,0);
    for(int i=1;i<n+1;i++){
        if(i==n){
            if(weight[i-1]<=w) 
            dp[w]=max(dp[w],pre[w-weight[i-1]]+value[i-1]);
            break;
        }
        for(int j=1;j<w+1;j++){
            if(weight[i-1]<=j) 
            dp[j]=max(dp[j],pre[j-weight[i-1]]+value[i-1]);
        }
        pre=dp;
    }
    cout<< dp[w] <<endl;
}
\end{lstlisting}



\end{homeworkProblem}
\begin{homeworkProblem}

\vspace{4pt}
\textbf{\large{The Shortest Path Problem}}

\vspace{4pt}

The state transition equation is as below:
\begin{equation}
    \begin{split}
        dst[u]=max \left\{ dst[u],dst[v]+w \right\}
    \end{split}
\end{equation}
Since the question descprtion tells that it contains negative edge, we adopt Bellmanford to solve this question.
We use two-demision array $edges$ to save edge with weight, and one-demision array $dst$ to represent the shortest path from start point to
each station. Bellmanford claims that the shortest path will not cover than n-1 edges. Therefore in the i-th loop, dst[u] represent the shortest distances to u if 
only go through i edges.
\vspace{4pt}

Hence the optimal method just takes \textbf{$O(n^2)$} space complexity and  \textbf{$O(n^2)$}
time complexity.

The cpp code is attached as follow:
\begin{lstlisting}
#include<iostream>
#include<vector>
#include<cmath>
#include<climits>
using namespace std;
int main(){
    int n,m,s,t;
    cin>>n>>m>>s>>t;
    vector<vector<int>> edges;
    for(int i=0;i<m;i++){
        int u,v,w;
        cin>>u>>v>>w;
        edges.push_back({u,v,w});
    }
    vector<int> dst(n+1,INT_MAX);
    dst[s]=0;
    for(int i=0;i<n-1;i++){
        int flag=0;
        for(const auto& e:edges){
            int u=e[0];
            int v=e[1];
            int w=e[2];
            if(dst[u]!=INT_MAX && dst[v]>dst[u]+w){
                dst[v]=dst[u]+w;
                flag=1;
            }
        }
        if(!flag) break;

    }
    cout<<dst[t]<<endl;
}
\end{lstlisting}
\end{homeworkProblem}
\begin{homeworkProblem}

    \vspace{4pt}
    \textbf{\large{LIS}}
    
    \vspace{4pt}
    
    Different from what we learn in class, I optimalize the algirithm with less memory and time cost. In this algorithm, we just creat one-demision array
    to save increasing sequence. In the i-th loop, if $nums[i] > back \ of \ dp$ put the nums[i] into the end of the array. Otherwise find the one which is larger than
    nums[i] from left, and replace it.
    \vspace{4pt}
    
    Hence the optimal method just takes \textbf{$O(n)$} space complexity and  \textbf{$O(nlogn)$}
    time complexity.
    
    The cpp code is attached as follow:
    \begin{lstlisting}
#include<iostream>
#include<vector>
using namespace std;
int main(){
    int n;
    cin>>n;
    vector<int> nums;
    int num;
    for(int i=0;i<n;i++) {
        cin>>num;
        nums.push_back(num);
    }
    if (n <= 1) return n;
    vector<int> dp;
    dp.push_back(nums[0]);
    for (int i = 1; i < n; ++i) {
        if (dp.back() < nums[i]) {
            dp.push_back(nums[i]);
        } else {
            /* 找到第一个比nums[i]大的元素 */
            auto itr = lower_bound(dp.begin(), dp.end(), nums[i]);
            *itr = nums[i];
        }
    }
    cout<<(int)dp.size()<<endl;
}
    \end{lstlisting}
    \end{homeworkProblem}

    \begin{homeworkProblem}

        \vspace{4pt}
        \textbf{\large{Max M Sum Subsequences Problem}}
        
        \vspace{4pt}

        The state transition equation is as below:
        \begin{equation}
            \begin{split}
                DP[i,j]=
                \left\{
                \begin{array}{lll}
                        mk[i-1,j-1]+nums[j], \ \ i==j \\
                        max(DP[i,j-1]+nums[j],mk[i-1,j-1]+nums[j]), \ \ else.
                \end{array}
                \right.
            \end{split}
        \end{equation}
        
        In above equation, $DP[i,j]$ represents max i sum subsequnces  of j preflix of sequence when the j-th number is in
        the i-th subsequence, and $mk[i,j]$ represents max i sum subsequnces of j preflix of the sequence. In order to compress 
        states, we also adopt scrolling arrays to replace above two-demision arrays.
        \vspace{4pt}
        
        Hence the optimal method just takes \textbf{$O(n)$} space complexity and  \textbf{$O(nlogn)$}
        time complexity.
        
        The cpp code is attached as follow:
    \begin{lstlisting}
#include<iostream>
#include<vector>
#include<cmath>
#include<cstring>
using namespace std;
int main(){
    int n,m;
    cin>>n>>m;
    int dp[n];
    int nums[n];
    // vector<int> nums(n,0);
    // vector<int> dp(n,0);
    for(int i=0;i<n;i++){
        int num;
        cin>>num;
        nums[i]=num;
    }
    int res;
    int max_sum;
    // vector<int> mk(n,0);
    int mk[n];
    memset(dp,0,sizeof(int)*n);
    memset(mk,0,sizeof(int)*n);
    for(int i=0;i<m;i++){
        max_sum=INT_MIN;
        for(int j=i;j<n;j++){
            if(j==i) {
                if(j==0) dp[j]=nums[j];
                else dp[j]=mk[j-1]+nums[j];
            }
            else {
                dp[j]=max(dp[j-1]+nums[j],mk[j-1]+nums[j]);
                mk[j-1]=max_sum;
            }
            max_sum=max(max_sum,dp[j]);
        }
    }
    cout<<max_sum<<endl;

}
    \end{lstlisting}
        \end{homeworkProblem}
\end{document}

